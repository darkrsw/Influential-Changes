\begin{table}[!t]
    \caption{Features used in learning a prediction model for ICs.}
    \label{tab:learningfeatures}
    \centering
\resizebox{\linewidth}{!}{
\begin{tabular}{l|l|c|l}
Category & Feature & Type & Description \\ \hline
\multirow{3}{*}{Structural} & \# files & Numeric & Number of files simultaneously changed in a single commit. \\ \cline{2-4}

& \# l-added & Numeric & Number of lines added to the program repository by the commit. \\ \cline{2-4}

& \# l-removed & Numeric & number of lines removed from the code base by the commit. \\ \hline

\multirow{3}{*}{NL Terms} & \# freq($token_{i}$) & Numeric & Frequency of occurrence of $token_{i}$. The number of features varies for each project.\\ \cline{2-4}

& Subjectivity & Boolean & Binary value indicating whether a commit message is subjective or objective. \\ \cline{2-4}

& Polarity & Boolean & Binary value indicating whether a commit message is negative or positive \\ \hline

\multirow{4}{*}{Co-change} & Max. PageRank & Numecric & Maximum value of PageRank for each file in a commit. \\ 
& & & PageRank is computed in a co-change graph built on a revision history of a project. \\ \cline{2-4}

 & Min. PageRank & Numeric & Minimum value of PageRank for each file in a commit. \\ \cline{2-4}
 

  & Betweeness Centrality & Numeric & Delta of the \textit{betweenness centrality} values of files between current and previous commits. \\ \cline{2-4}
  
  & Closeness Centrality & Numeric & Delta of the \textit{closeness centrality} values of files between current and previous commits. \\

\bottomrule

\end{tabular}
}

\end{table}