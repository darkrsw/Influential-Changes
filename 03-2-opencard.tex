%\begin{sidewaysfigure}[ht]
\begin{landscape}
 \begin{figure}
\centering
\includegraphics[width=\linewidth]{fig/response-boxplot.pdf}
\caption{Survey results on different categories of Influential Changes.}
\label{fig:survey}
%\end{sidewaysfigure}
 \end{figure}
\end{landscape}


\subsection{Developer Validation}
\label{subsec.opencard}

To further validate the influential software changes dataset that we have
collected with our intuition-based post-mortem metrics, we perform a large-scale developer study. Instead of asking developers to confirm each identified
commit, we must summarize the commits into categories. To that end, we
resorted to open-card sorting~\cite{Nielsen95}, a well known, reliable and
user-centered method for building a taxonomy of a system~\cite{boxesandarrows}.
Card sorting helps explore patterns on how users would expect to find content
or functionality. In our case, we use this technique to label influential
software changes within categories that are easily differentiable for
developers.

We consider open-card sorting where participants are given cards showing
description of identified influential software changes\footnote{We consider
all 177 influential software changes from the post-mortem analysis.} without
any pre-established groupings. They are then asked to sort cards into groups
that they feel are acceptable and then describe each group. We performed this
experiment in several iterations: first two authors of this paper provided
individually their group descriptions, then they met to perform another
open-card sorting with cards containing their group descriptions. Finally, a
third author, with more experience in open-card sorting, joined for a final
group open-card sorting process which yielded 12 categories of influential
software changes.

The influential software changes described in the 12 categories span over
four software maintenance categories initially defined by Lientz {\em et
al.}~\cite{Lientz:1978:CAS:359511.359522} and updated in ISO/IEC 14764. Most
influential software changes belong to the {\em corrective changes} category.
Others are either {\em preventive changes}, {\em
adaptive changes} or {\em perfective changes}. Finally, changes in one of our influential change categories
 can fall into more than one maintenance categories. We refer to
them as {\em cross area changes}.

\textbf{Developer assessment.}
We then conduct a developer survey to assess the relevance of the 12 categories of influential changes that we describe. 
The survey participants have been selected from data collected in the GHTorrent project~\cite{Gousi13} which contains history
archives on user activities and repository changes in GitHub. We consider active developers (i.e., those who have contributed in the latest changes recorded in GHTorrent) and focus on those who have submitted comments on other's commit. We consider this to be an indication of experience with code review. The study\footnote{Survey form at \url{https://goo.gl/V2g8OE}} was sent to over 1952 developer email addresses. After one week waiting period, only 800 email owners opened the mail and 144 of them visited the survey link. Finally 89 developers volunteered to participate in the survey. 66 (i.e., 74\%) of these developers hold a position in a software company or work in freelance. Nine respondents (10\%) are undergraduate students and eight (9\%) are researchers. The remaining six developers did not indicate their current situation. In total, 78\% of the participants confirmed having been involved in code review activities. 26 (29\%) developers have between one and five years experience in software development. 29 (33\%) developers have between five and ten years of experience. The remaining 34 (38\%) have over ten years of experience. 

In the survey questionnaire, developers were provided with the name of a category of influential software changes, its description and an illustrative example from our dataset (we provided the same example for each category to every participant). The participant was then requested to assess the relevance of this category of changes as influential software using a Likert scale between {\tt 1:very influential} and {\tt 5:unimportant}. Figure~\ref{fig:survey} summarizes the survey results. For more detailed description of the categories, we refer the reader to the project web site (see Section ``Availability'').

The survey results suggest that:
\begin{itemize}
	\item According to software developers with code review experience, all 12 categories are about important changes: 7 categories have an average agreement of 2 (i.e., Influential), the remaining 5 categories have an average of 3 (i.e., potentially influential). Some (e.g.,  ``domino changes'' and ``changes fixing pervasive bugs'') are clearly found as more influential than others (e.g., ``important test case addition'').
	\item Some changes, such as ``fixes for hard to reproduce or locate bugs'', are not as influential as one might think.
	\item Developers also suggested two other categories of influential changes: {\em Documentation changes} and {\em Design-phase changes}. The latter however are challenging to capture in source code repository artefacts, while the former are not relevant to our study which focuses on source code changes.
\end{itemize}

With this study we can increase our confidence in the dataset of influential
software changes that we have collected. We thus consider leveraging on the
code characteristics of these identified samples to identify more influential
changes.
