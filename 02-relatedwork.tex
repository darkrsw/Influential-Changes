\section{Related Work}
\label{sec:related}

This section discusses four groups of related work; 1) software evolution, 2)
change impact analysis, 3) defect prediction, and 4) developer expertise. These
topics address several relevant aspects of our study.

\subsection{Software Evolution}
Changing any file in a software system implies that the system evolves in a
certain direction. Many studies dealt with software evolution in different ways.
D'Ambros et al.~\cite{dambros_evolution_2006} presented \emph{the evolution
radar} that visualizes file and module-level coupling information. Although this
tool does not directly predict or analyze the change impact, it can show an
overview of coupling relationships between files and modules.
Chronos~\cite{servant_history_2012} provides a narrowed view of history
slicing for a specific file. The tool analyzes a line-level history of a file.
This reduces the time required to resolve program evolution tasks. Girba et
al.~\cite{girba_how_2005} proposed a metric called \emph{code ownership} to
illustrate how developers drive software evolution. We used the metric to
examine the influence of a change.

API evolution and deprecation, which often leads to important collateral evolutions
can be influential in a project development. Several studies~\cite{Dagenais:2009:SAR:1555001.1555083,DBLP:conf/sigsoft/RobbesLR12,DBLP:conf/icsm/SawantRB16,Teyton:2014:SLM:2926316.2926322,Kim:2010:MAI:1831508.1831654,Dagenais:2011:RAC:2000799.2000805,6062100} in the literature have
investigated such evolution. For example, Dagenais and Robillard have proposed SemDiff~\cite{Dagenais:2009:SAR:1555001.1555083},
a framework for recommending replacements of non-trivial changes. Robbes et al.~\cite{DBLP:conf/sigsoft/RobbesLR12}
have studied the importance in practice of API deprecation for developers. Teyton et al.~\cite{Teyton:2014:SLM:2926316.2926322} have built a framework for analyzing software library dependencies to support developers in library migrations based on evolution trends.



\subsection{Change Impact Analysis}

\dongsun{we need to discuss~\cite{shihab_industrial_2012} and~\cite{shihab_high-impact_2011}.}

Many previous studies revealed a potential impact of software changes. There is a
set of techniques that use dynamic analysis to identify change impacts. Ren et
al.~\cite{ren_chianti:_2004} proposed \emph{Chianti}. This tool first runs test
cases on two subsequent program revisions (after/before a change) to figure out
atomic changes that describe behavioral differences. The authors provided a
plug-in for Eclipse, which help developers browse a change impact set of a
certain atomic change. FaultTracer~\cite{zhang_faulttracer:_2012} identifies
a change impact set by differentiating the results of test case executions on
two different revisions. This tool uses the extended call graphs to select
test cases affected by a change.

Brudaru and Zeller~\cite{brudaru_what_2008} pointed out that the long-term
impact of changes must be identified. To deal with the long-term impact, the authors proposed
a change genealogy graph, which keeps track of dependencies between
subsequent changes. Change genealogy captures addition/change/ deletion of
methods in a program. It can measure long-term impact on quality,
maintainability, and stability~\cite{herzig_capturing_2010}. 
In addition, it can reveal cause-effect chains~\cite{herzig_mining_2011} and predict
defects~\cite{herzig_predicting_2013}.


Although dynamic analysis and change genealogy can pinpoint a specific element
affected by a change in source code, its scope is limited to executed statements
by test cases. This can miss many affected elements in source code as well as
non-source code files such as build scripts and configuration settings.
Revision histories can be used for figuring out files changed frequently
together. Zimmermann et al.~\cite{zimmermann_mining_2004} first studied
co-change analysis in which the authors revealed that some files are
commonly changed together. Ying et al.~\cite{ying_predicting_2004} proposed an
approach to predicting files to change together based on revision histories.


There have been cluster-based techniques for change impact analysis. Robillard
and Dagenais~\cite{robillard_retrieving_2008} proposed an approach to building
change clusters based on revision histories. Clusters are retrieved by analyzing
program elements commonly changed together in change sets. Then, the approach
attempts to find matching clusters for a given change. The matching clusters are
regarded as the change impact of the given change. Sherriff and
Williams~\cite{sherriff_empirical_2008} presented a technique for change impact
analysis using singular value decomposition (SVD). This technique basically
figures out clusters of program elements frequently changed together. When clustering
changes, the technique performs SVD. The clusters can be used for identifying
the change impact of an incoming change. 
 


%AOP-based. Necessary here?
% Liu et al.~\cite{liu_change_2011} presented an approach to identify change
% impacts in aspect-oriented programs. This approach analyzes
% potential control-flow changes to search for program dependencies when the
% program is woven by aspects.

\subsection{Defect Prediction}

Changing a program may often introduce
faults~\cite{sliwerski_hatari:_2005,kim_automatic_2006}.
Thus, fault prediction at an early stage can lead developers to achieving
a better software quality. Kim et al.~\cite{kim_predicting_2007} proposed a
cache-based model to predict whether an incoming change may introduce or not.
They used \emph{BugCache} and \emph{FixCache} that record entities and files
likely to introduce a bug and fix the bug if they are changed. The results of
their empirical study showed that the caches 46-95\% accuracy in seven open
source projects.

Machine learning classification can be used for defect prediction as well. Kim
et al.~\cite{kim_classifying_2008} presented an approach to classifying
software changes into buggy or clean ones. They used several features such as number of
lines of added/deleted code, terms in change logs, and cyclomatic complexity.
The authors conducted an empirical evaluation on 12 open source projects. The
result shows 78\% prediction accuracy on average. In addition, Shivaji et
al.~\cite{shivaji_reducing_2009} proposed a feature selection technique to
improve the prediction performance of defect prediction. Features are not
limited to metrics of source code; Jiang et al.~\cite{jiang_personalized_2013}
built a prediction model based on individual developers. Defect prediction
techniques are often faced with imbalanced datasets. Bird et
al.~\cite{bird_fair_2009} pointed out that unfair and imbalanced datasets can
lead to bias in defect prediction.


\subsection{Developer Expertise}

It is necessary to discuss developer expertise since influential changes implies
that the developer who made the changes can be influential to other developers.

As the size of open-source software projects is getting larger, developer
networks are naturally constructed and every activity in the network may affect
other developers substantially. Hong et al.~\cite{hong_understanding_2011}
reported a result of observing a developer social network. The authors
investigated Mozilla's bug tracking site to construct a developer social network
(DSN). In addition, they collected general social networks (GSNs) from ordinary
social communities such as Facebook and Amazon. This paper provides the
comparison between DSN and GSNs. Findings described in this paper include 1) DSN
does not follow power law degree distribution while GSNs do, 2) the size of
communities in DSNs is smaller than that of GSNs. This paper also reports the
result of evolution analysis on DSNs. DSNs tend to grow overtime but not much as
GSNs do.

Onoue et al.~\cite{onoue_study_2013} studied and enumerates developer activity
data in \texttt{Github.com}.
It classifies good developers, tries to understand developers, and
differentiates types of developers. However, the paper does not provide any
further implication. In addition, there is no result for role analysis and 
social structure.

Pham et al.~\cite{pham_creating_2013} reported the results of a user study which
has been conducted to reveal testing culture in OSS. The authors have
interviewed 33 developers of GitHub first and figured out the transparency of
testing behaviors. Then, an online questionnaire has been sent to 569 developers
of GitHub to find out testing strategies.
