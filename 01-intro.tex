\section{Introduction}
\label{sec:intro}

Current development practices heavily rely on version control systems to
record and keep track of changes committed in project repositories. While many
of the changes may be simply cosmetic or provide minor improvements, others
have a wide and long-term influence to the entire system and related systems.
Brudaru and Zeller~\cite{brudaru_what_2008} first illustrated examples of 
changes with long term-influence: 1) changing access privilege
(i.e., \texttt{private} $\rightarrow$ \texttt{public}), 2) changing kernel
lock mechanism, and 3) forgetting to check a null return. If we can predict
whether an incoming software change is influential or not, either positively
or negatively, just after it is committed, it could significantly improve
maintenance tasks (e.g., easing debugging if a new test harness is added) and provide insights for recommendation systems (e.g., code reviewers can focus on fewer changes).


The influence of a software change can however be hard to detect immediately since it often does
not involve immediate effects to other software elements. Instead, it can
constantly affect a large number of aspects in the software over time. Indeed,
a software change can be influential not only inside and/or beyond the project
repository (e.g., new defects in code base and new API calls from other
programs), but also immediately and/or  long after the changes have been
applied. The
following are examples of such influential changes:

{\bf Adding a new lock mechanism}: {\tt mutex-lock} features were introduced in 
Linux 2.6 to improve the safe execution of kernel critical code sections. However,
after their introduction, the defect density of Linux suddenly increased for 
several years, largely contributed by erroneous usage of these features.
Thus, the influence of the change was not limited to a specific set of modules.
Rather, it was a system-wide problem. 
%Unfortunately, the developers realized
%later that the new feature had a wide influence on the system (see more
%details in Section~\ref{sec:motivation2}).

{\bf Changing build configurations}: A small change in configuration files may
influence the entire program. In \texttt{Spring-framework}, a developer missed
file inclusion options when migrating to a new build system
(\texttt{$\ast$.aj} files were missing in \texttt{build.gradle}\footnote{Commit \tt\small
a681e574c3f732d3ac945a1dda4a640ce5514742}\footnote{Bug report \url{https://jira.spring.io/browse/SPR-9576}}). 
This makes an impact since programs depending on the framework failed occasionally
to work. The reason of this failure (missed file) was hard to pinpoint. This bug 
has been fixed after six months after its introduction.
%This did not
%have an immediate impact but it could make a long-term influence to the framework as
%well as other programs depending on the framework. \david{Why ? It is best to say that it ``makes an impact" and the impact is X, Y, Z ...}
%This could not be detected by
%dynamic analysis or techniques limited to the impact of source code changes (see
%more details in Section~\ref{sec:isolated}).

{\bf Improving performance for a specific environment}:
\texttt{FastMath.floor()} method in \emph{Apache Commons Math} had a problem
with Android applications since it has a static code block that makes an
application hang about five seconds at the first call. Fixing this issue
improves the performance of all applications using the library. 
%This influence
%can be revealed after many client applications upgrade the dependency
%information (i.e., version-up) (see more details in
%Section~\ref{sec:blocking}).

% Analyzing how software changes affect the program is one of the most important
% tasks in software maintenance since changes can affect the quality,
% maintainability, and effort of software systems~\cite{brudaru_what_2008}. This
% analysis can reveal program elements (e.g., methods and types), relevant
% artifacts (e.g., configuration files), and developers that a certain change may
% affect.


\vspace{0.15in}
Unfortunately, existing techniques are limited to revealing the \emph{short-
term impact} of a certain software change. The short-term impact indicates an
immediate effect such as test case failure or coverage deviation. For example,
dynamic change analysis
techniques~\cite{ren_chianti:_2004,zhang_faulttracer:_2012} leverage coverage
metrics after running test cases. Differentiating coverage information
before/after making a change shows how the change influences other program
elements. Other approaches are based on similarity distances~\cite{robillard_retrieving_2008,sherriff_empirical_2008}.
 These
firstly identify clusters of program elements frequently changed
together or tightly coupled by analyzing revision histories. Then, they
attempt to figure out the best-matching clusters for a given change.
Developers can assume that program elements (e.g., files or methods) in the
cluster may be affected by the given change. Finally, change genealogy~\cite{herzig_capturing_2010,herzig_mining_2011,herzig_predicting_2013} 
approaches keep track of dependencies between subsequent changes, and can capture some long-term impact of changes.
However, it is limited to identifying source code entities and defect density. Overall, all  the above
techniques may not be successful in predicting a wide and long-term influence
of software changes. This was unfortunately inevitable since those existing
techniques focus only on explicit dependencies such as method calls.

\paragraph*{\bf Definition scoping for Influential Changes} -- As hinted by the examples above, 
an influential change can occur in different part of the code base, and may relate to a variety
of concepts (architecture, API, or algorithm implementations). We scope the definition of an influential
change in this paper as a {\em source code change} which may induce significant (i.e., visible) change 
in the behaviour/evolution of any of the three entity in the ecosystem of project development: the code base, the end-users, or the developers. Thus, we consider the three following definitions in this paper:
\begin{itemize}
	\item {\bf $[$Code base$]$} - An influential change is any change that will {\em motivate/necessitate other changes} in the code base. Such a change affects the internal development processes of code review, testing, etc. Examples of such changes include domino changes that are necessary to perform collateral evolution when a popular API is invasively modified.
	\item {\bf $[$User base$]$} - An influential change is any change that eventually {\em impacts software adoption or its use}.  Such a change can occur in various circumstances, as part of specific effort to attract/retain users (e.g., change that focus on improving library API usability), or inadvertently with user-undesired changes (e.g., invasive change in key feature).
	\item {\bf $[$Developer team$]$} - An influential change is any change that {\em impacts the dynamics among developers}. Such a change can positively impact the development team by stimulating development (e.g., a fix of a blocking bug~\cite{ValdiviaGarcia:2014:CPB:2597073.2597099}), require developers to focus on re-learning about how the code is built (e.g., a major structural change), or address a standing issue that left the team in disagreement (e.g., reverting a controversial change).
\end{itemize}

We recognize that these definitions are not exhaustive for identifying all changes that are potentially ``influential''. Nevertheless, within the scope of this paper, they allow for investigations on the risk of a code change, a topic that is still poorly studied in the literature.
\paragraph*{\bf Study Research Questions} -- In this study we are interested in investigating the following research questions:
\begin{itemize}
\itemsep0em
	\item[RQ1:] What constitutes an influential software change? Are there developer-approved definitions/descriptions of influential software changes?
	\item[RQ2:] What metrics can be used to collect examples of influential software changes?
	\item[RQ3:] Can we build a classification model to identify influential software changes immediately after they are applied?
\end{itemize}

% as the quality score that measures the defect
% density of a software change on the change genealogy~\cite{herzig_predicting_2013}. The long-term impact can be used
% for predicting defects~\cite{herzig_predicting_2013} and identifying
% causal-effect relations from version histories~\cite{herzig_mining_2011}.

\paragraph*{\bf Terminology} -- In this work we use several terms that require explicit definitions to avoid confusion. The following are the most recurrent terms:

{\em post-mortem analysis}: we use this expression to refer to investigations of changes whose impact on the software ecosystem has become apparent (e.g., a change that stops all discussion on a specific issue).

{\em anomaly}: we use this term to refer to a metric whose value is out of the ordinary. It is determined via outlier detection in a series of measurements. 
% An anomaly is identified regardless of whether it is exceptional or not. 
Indeed, abnormal changes can be numerous in a project.

{\em change behaviour}: we use this expression to refer to the the observable metrics associated to a series of changes in a software project. For example, the average delay between two commits (i.e., rhythm of commit) is part an example of change behaviour observation.

{\em prediction}: we use this term to refer to a classification decision, based on machine learning methods, on whether a given commit will eventually be revealed as influential. The objective of a prediction is to ``know'' now whether a given change will become apparent far later, based on observations of previous changes.
 

\paragraph*{\bf This Study} -- To automatically figure out whether an incoming software change will be influential, we
designed a classification technique based on machine learning. Since
the technique requires labeled training instances, we first discovered existing
influential changes in several open source projects in order to obtain baseline
data. Specifically, we collected \numChanges code commits from \numSubjects open source projects and did
post-mortem analysis to identify influential changes. This analysis examined
several aspects of influential changes such as controversial changes and breaking
behaviors. In addition, we manually analyzed whether those changes actually have
long-term influence to revision histories.
As a result, we could discover several influential changes from each subject. Note that change properties that present obvious patterns for recognizing a change as influential with high probabilities are post-mortem properties: the influence of the commit has been established at this point (e.g., impact on user adoption is now visible).
We further label these changes to build category definition for influential software changes
through an open-card sorting process. These categories are then validated by
developers with experience in code review.   


% Furthermore, we extracted \numFeatures features such as \dongsun{list up some
% features here} to  Then, we designed the change influence model to [...]

Based on the influential changes we discovered in the above study, we extracted
feature vectors for machine-learning classification. These features include program structural
metrics~\cite{kim_classifying_2008}, terms in change
logs~\cite{kim_classifying_2008}, and co-change metrics~\cite{Beyer05}.
Then, we built a classification model by leveraging machine learning algorithms such as
Na\"{i}ve Bayes~\cite{lewis:ecml:1998,mlbook} and Random
Forest~\cite{breiman_random_2001}. The prediction/classification is about determining when a change is being committed whether it could be influential (later). We leverage machine learning to build a model based on features available at the time of prediction/classification, but that would not be used to measure ``influence'' per se. 
To evaluate the effectiveness of this technique, we conducted experiments that
applied the technique to \numSubjects projects.
Experimental assessment results with a representative, randomly sampled, subset of our data show that 
our classification model achieves good performance. Given a change being applied, our classifier is able
to classify it as ``influential'' with a precision of 
\precisionwild. The classifier is further able to identify \recallwild (recall) of the influential changes 
in the dataset. The overall F-measure of the classifier is thus \fmeasurewild.

This paper makes the following contributions:

\begin{itemize}
\itemsep0em 
	\item Collection of influential software changes in popular open source projects.
	\item Definition of influential software change categories approved by the software development community.
	\item Correlation analysis of several program metrics and influential software
changes.
	\item Accurate machine-learning classification model for influential software changes.
	%\item Evaluation of our prediction model by using \numSubjects open source
	%projects.
\end{itemize}

%\dongsun{do not miss reorganizing this paragraph.}
The remainder of this paper is organized as follows. After describing motivating
examples in Section~\ref{sec:motivation}, we present our study results of
post-mortem analysis for discovering influential changes in
Section~\ref{sec:preliminary}.
Section~\ref{sec:model} provides our design of a classification model for
influential changes together with a list of features extracted from software changes.
In addition, the section reports the evaluation result of experiments in which
we applied the classification model to open source projects.
Section~\ref{sec:discussion} discusses the limitations of our work as well as threats to validity.
After surveying the related work in Section~\ref{sec:related}, 
we conclude with directions for future research in
Section~\ref{sec:conclusion}.
