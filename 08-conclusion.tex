\section{Conclusion and Future Work}
\label{sec:conclusion}

In software revision histories, we can find many cases in which a few lines of
software changes can positively or negatively influence the whole project
while most changes have only a local impact. In addition, those
\emph{influential changes} can constantly affect the quality of software for a
long time. Thus, it is necessary to identify the influential changes at an
early stage to prevent project-wide quality degradation or immediately
take advantage of new software new features.

In this paper, we reported results of a post-mortem analysis on \numChanges
software changes that are systematically collected from \numSubjects open
source projects  and labelled based on key quantifiable criteria. We then
used open-card sorting to propose categories of influential changes.
After developer have validated these categories, we consider examples of
influential changes and extract features such as 
complexity and terms in change logs in order to build a classification model.
We showed that the classification features are efficient beyond the scope
of our initial labeled data on influential changes. 
Our future work will focus on the following topics: 

\begin{itemize}
  \item Influential changes may affect the popularity of projects. We will
  investigate the correlation between influential changes and popularity metrics
  such as the number of new developers and new fork events.
  \item In our study, we used only metrics for source code. However, features of
  developers can have correlations with influential changes. We will study
  whether influential changes can make developer influential and vice versa.
  \item Once influential changes are identified, it is worth finding out who can
  benefit from the changes. Quantifying the impact of the influential changes to
  developers and users can significantly encourage further studies.
\end{itemize}
